\chapter{Zusammenfassung und Ausblick}
Das Buchungssystem besteht aus einer clientseitigen Anwendung und einer serverseitigen API. Über die API können Nutzer mittels HTTP-Anfragen Buchungen erstellen, ändern und löschen. Die serverseitige Implementierung erfolgt mit Express und ist dadurch vereinfacht. Zur Sicherung der Buchungsdaten wird eine MongoDB-Datenbank verwendet mit die der Server kommuniziert. Für die Erstellung von Buchungen steht dem Nutzer auf der Client-Seite ein Buchungsdialog zur Verfügung, der mithilfe von den gängigen Front-End-Tools wie HTML, Less und JavaScript umgesetzt wurde. Die Umsetzung des Systems ist ohne großen Aufwand gelungen und erfüllt die Ziele der Arbeit. Insgesamt kann die Implementierung eines Buchungssystems unter Verwendung von Node.js eine robuste und skalierbare Lösung zur Verwaltung von Buchungen bieten. Darauf aufbauend könne man in die Buchung von Zimmern außerdem das Auswählen von Extras hinzufügen.
