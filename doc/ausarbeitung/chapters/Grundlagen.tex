\chapter{Grundlagen}

\section{HTML, CSS, JS}
\subsection{HTML}
\subsection{Less}
\subsection{JavaScript}

\section{Node.js}
Die serverseitige Grundlage des Systems bildet die Laufzeitumgebung Node.js. Mit dieser ist es möglich JavaScript Programme abseits eines Browsers ausführen zu können. Mithilfe von \glqq npm\grqq, dem Paketmanager von Node.js, werden außerdem externe Pakete für bestimmte Funktionalitäten verwendet um die Implementierung zu vereinfachen.

\subsection{Express}
Das Express-Framework ermöglicht eine unkomplizierte Erstellung einer Webanwendung. So wird beispielsweise die Verarbeitung von HTTP-Anfragen wird mithilfe von spezifischen Methoden deutlich vereinfacht und macht das Erstellen einer API besonders effizient. Innerhalb der Buchungssoftware sorgt Express für die Bereitstellung eines Servers, der alle clientseitigen Dateien zur Verfügung stellt und auf HTTP-Anfragen zur Erzeugung oder zum Abrufen von Buchungen, reagiert.

\subsection{Nodemailer}
Nodemailer ist ein \glqq npm\grqq-Paket, das dafür verwendet wird E-Mails mit Node.js zu versenden. Durch Angabe der Adresse von der die E-Mail verschickt werden soll und verschiedenen Optionen wie z.B. dem Betreff wird das Senden vereinfacht. Um eine E-Mail zu versenden muss im ersten Schritt ein \glqq Transporter\grqq zusammen mit dem verwendeten Mail Dienst und den Anmeldeinformationen der zu sendenen E-Mail Adresse, erzeugt werden. Anschließend werden die Mail-Optionen initialisiert und über die Methode \glqq sendMail\grqq des Transportes, die E-Mail versandt.

\subsection{Formidable}

\section{MongoDB}
Unter den dokumentenorientierten NoSQL-Datenbanken ist MongoDB die meist verwendete. Die wichtigsten Bestandteile einer MongoDB Datenbank sind Dokumente und Collections. Dokumente sind Datenstrukturen die sich aus einem oder mehreren Paaren, bestehend aus Feldern und dazugehörigen Werten, zusammenfügen. In ihrem Aufbau sind diese Dokumente identisch mit JSON Objekten. Collections hingegen sind Sammlungen von verschiedenen Dokumenten und sind vergleichbar mit Tabellen aus relationalen Datenbanken.
\\
\\
Damit eine Node.js-Anwendung mit einer MongoDB kommunizieren kann wird beispielsweise das \glqq npm\grqq-Paket \glqq mongodb\grqq benötigt. Hier können dann auf den einzelnen Collections, Methoden angwendet werden. Die wichtigsten Methoden für das zu erstellende System sind \glqq find\grqq , \glqq insertOne\grqq , \glqq deleteOne\grqq und \glqq deleteMany\grqq. Mithilfe von find werden alle Dokumente einer Collection geliefert, auf die ein in der Methode spezifiziertes Kriterium zutrifft. InsertOne ermöglicht es ein neues Dokument in eine Collection hinzuzufügen und deleteOne bzw. deleteMany ermöglicht das Löschen von Dokumenten aus einer Collection.
