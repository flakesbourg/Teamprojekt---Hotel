\kurzfassung

Das Thema der Ausarbeitung ist das Erstellen einer Webseite für das fiktive Grandline Hotel mit Standort in der Vulkaneifel. Die Webanwendung soll dem Unternehmen ermöglichen, Zimmerbuchungen online anzubieten. Dadurch soll der Prozess des Buchens oder Abänderns einer bereits getätigten Buchung erleichtert und dem Kunden alle nötigen Informationen dargeboten werden. Ziel ist es, den Umsatz des Unternehmens durch mehr Buchungen zu steigern.
\\
\\ 
Zuerst müssen die Grundlagen für den client- und den serverseitigen Teil der Anwendung gelegt werden. Dazu zählen die Grundlagen von HTML, Less, JavaScript/Node.js usw. Anschließend werden einige Konzepte dargestellt und die Implementierung der Anwendung beschrieben. Um die Buchung umzusetzen, wird eine API mithilfe von Express in Node.js implementiert. Es wird außerdem eine beispielhafte Benutzung der Anwendung dargestellt und weitere Szenarien aufgezeigt, in denen man ein solches Buchungssystem gebrauchen kann. Schlussendlich wird die Arbeit zusammengefasst und ein Ausblick für weitere Verbesserungen gegeben.

\kurzfassungEN

The topic of the paper is the creation of a website for the fictional Grandline Hotel located in the Vulkaneifel region. The web application should enable the company to offer room bookings online. This should make the process of booking or changing an already made reservation easier and provide the customer with all necessary information. The goal is to increase the company's revenue through more bookings.
\\
\\
First, the fundamentals for both the client-side and server-side parts of the application need to be established. This includes the basics of HTML, Less, JavaScript/Node.js, and so on. Afterwards, some concepts will be presented and the implementation of the application will be described. To implement the booking, an API will be implemented using Express in Node.js. Additionally, an exemplary usage of the application will be demonstrated, and other scenarios in which such a booking system could be useful will be presented. Finally, the work will be summarized, and an outlook for further improvements will be provided.
